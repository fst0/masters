\documentclass[10pt]{article} % Default font size is 12pt, it can be changed here
\usepackage{amsmath}

\usepackage{geometry} % Required to change the page size to A4
\geometry{a4paper} % Set the page size to be A4 as opposed to the default US Letter

\usepackage{graphicx} % Required for including pictures

\usepackage{float} % Allows putting an [H] in \begin{figure} to specify the exact location of the figure

\linespread{1.2} % Line spacing
\graphicspath{{Pictures/}} % Specifies the directory where pictures are stored

\begin{document}

\section{Magnetic Domain Structures} % Major section
\subsection{Magnetic Domains}
Pierre Weiss stated in 1907 that a ferromagnet possesses a number of small regions("magnetic domains). Each of them exhibits the saturation magnetization. It is important that the magnetization direction of single domains each along the easy axis are not necessarily parallel. Domains are separated by domain boundaries or domain walls.

\begin{itemize}
\item In soft materials small external fields ($\approx 10^{-6} T$) are sufficient to reach saturation magnetization ($\mu_0 M \approx 1T$). The external field need not order all magnetic moments macroscopically (because in each domain they are already ordered) but has to align the domains. Thus, a movement of domain walls only occurs which requires low energy.

\item It is possible that ferromagnetic materials exhibit a vanishing total magnetization $\vec{M} = 0$ below the critical temperature without applying an external field. In this situation each domain still possesses a  saturated magnetization but due to the different orientations the total magnetization is equal to 
\end{itemize}

\subsection{Magnetization of an Ideal & Real Crystal}




\begin{figure}[H]
\begin{center}
\includegraphics[scale=1.0]{hyloop}
\caption{Hysteresis Loop}
\end{center}
\end{figure}



\end{document}