
\documentclass[letterpaper,11pt]{texMemo} % Set the paper size (letterpaper, a4paper, etc) and font size (10pt, 11pt or 12pt)
\usepackage{parskip} % Adds spacing between paragraphs
\setlength{\parindent}{5pt} % Indent paragraphs

%----------------------------------------------------------------------------------------
%	MEMO INFORMATION
%----------------------------------------------------------------------------------------
%----------------------------------------------------------------------------------------

\begin{document}

\title(Changing the target in sputter system)

1. Drain the water(by...)

2. Vent the system by... Pressure about 700 . This takes about 20min, but you can continue.

3. Disconnect the power cables from the sample chamber. There are two: a DC power one that is pin based and a black cable.

4. Remove the water tube from the sample chamber. They are blue and connect all the sample chambers together in a ring.

5. Wait for the chamber to be fully vented. 

6. Main Chamber:
		Backing: Closed
		Vent: Open

	Load Lock:
		Backing: Open
		Vent: Closed


	pressure set point ~ $10^-3$ Torr

7. Take the gas inlet off and throw away the VCR gasket. They should not be reused.

8. Close Main Chamber Vent

9. Physically remove the sample chamber by removing the screws and pull the chamber out. 

10. By removing the shutter pin (big and white) move the shutter out of the way of the target

11. Removing the gasket surrounding the target by loosening the pins.

12. Remove the plate sitting ontop of the target by loosening the screws.

13. Remove the target. If it is magnetic, remove it by removing one of the screws and sliding it out. do not try to pull it out.
	
	If the target has any cracking, then the energy being used is too high.

14. Clean everything with isopropyl and look for small flakes of metal that need to be removed. If they are magnetic, use a magnetic screwdriver to remove 		them.

15. Place a new target on it and fasten it with the plate.

16. Check for short a short between the copper and the target. Also check for low resistance between the target and the pin with the RF shielding around 		it.

17. On the gasket there are two slots: one of large targets(0.25") and thin targets(0.125"). The thick targets use a slot in the shape of an "I", the thin 		ones use the other.

Place the gasket back on and into the correct slot and make sure to leave a little space between it and the target or else it will short. Then check for a 		short between the gasket and the target.

18. Place the shutter pin back in and set it to the closed position(brown lead), then move the shutter to the closed position with your hand and tighten 		the screw. This makes sure the "closed" position is really closed.

19. Remove the large copper gasket ring and put a new one on. They cannot be reused.

20. Now, put the sample chamber back into the machine. Make sure the groves on the machine and sample chamber line up. There is also a sticker with the 		chamber number on the sample chamber that indicated the correct way it must face.

21. Reconnect the gas inlet, do not forget to replace the VCR gasket.

22. Reconnect the rest of the cables and water lines. Also refill the water.
%----------------------------------------------------------------------------------------

\end{document}